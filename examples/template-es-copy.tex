%% Inicio del archivo `template-es.tex'.
%% Copyright 2006-2013 Xavier Danaux (xdanaux@gmail.com).
%
% This work may be distributed and/or modified under the
% conditions of the LaTeX Project Public License version 1.3c,
% available at http://www.latex-project.org/lppl/.


\documentclass[11pt,a4paper,sans]{moderncv}   % opciones posibles incluyen tamaño de fuente ('10pt', '11pt' and '12pt'), tamaño de papel ('a4paper', 'letterpaper', 'a5paper', 'legalpaper', 'executivepaper' y 'landscape') y familia de fuentes ('sans' y 'roman')

% temas de moderncv
\moderncvstyle{casual}                        % las opciones de estilo son 'casual' (por omision),'classic', 'oldstyle' y 'banking'
\moderncvcolor{blue}                          % opciones de color 'blue' (por omision), 'orange', 'green', 'red', 'purple', 'grey' y 'black'
%\renewcommand{\familydefault}{\sfdefault}    % para seleccionar la fuente por omision, use '\sfdefault' para la fuente sans serif, '\rmdefault' para la fuente roman, o cualquier nombre de fuente
%\nopagenumbers{}                             % elimine el comentario para suprimir la numeracion automatica de las paginas para CVs mayores a una pagina

% codificacion de caracteres
%\usepackage[utf8]{inputenc}                  % reemplace con su codificacion
%\usepackage{CJKutf8}                         % si necesita usa CJK para redactar su CV en chino, japones o coreano

% ajustes para los margenes de pagina
\usepackage[scale=0.75]{geometry}
%\setlength{\hintscolumnwidth}{3cm}           % si desea cambiar el ando de la columna para las fechas

% datos personales
\name{Alonzo}{Altamirano}
% \title{T\'itulo del CV (opcional)}                   % dato opcional, elimine la linea si no desea el dato
\address{San Francisco Bay Area, CA} % dato opcional, elimine la linea si no desea el dato
\phone[mobile]{+1~(925)~784~8786}                     % dato opcional, elimine la linea si no desea el dato
%\phone[fixed]{+2~(345)~678~901}                      % dato opcional, elimine la linea si no desea el dato
%\phone[fax]{+3~(456)~789~012}                       % dato opcional, elimine la linea si no desea el dato
\email{alonzoa@uoregon.edu}                                 % dato opcional, elimine la linea si no desea el dato
\homepage{Github: alonzno}                           % dato opcional, elimine la linea si no desea el dato
%\extrainfo{informacion adicional}                    % dato opcional, elimine la linea si no desea el dato
% \photo[64pt][0.4pt]{picture}                         % '64pt' es la altura a la que la imagen debe ser ajustada, 0.4pt es grosor del marco que lo contiene (eliga 0pt para eliminar el marco) y 'picture' es el nombre del archivo; dato opcional, elimine la linea si no desea el dato
\quote{Fresh computer science graduate seeking mentorship, experience, and challenge in the high-tech industry.}                       % dato opcional, elimine la linea si no desea el dato

% para mostrar etiquetas numericas en la bibliografia (por omision no se muestran etiquetas), solo es util si desea incluir citas en en CV
%\makeatletter
%\renewcommand*{\bibliographyitemlabel}{\@biblabel{\arabic{enumiv}}}
%\makeatother

% bibliografia con varias fuentes
%\usepackage{multibib}
%\newcites{book,misc}{{Libros},{Otros}}
%----------------------------------------------------------------------------------
%            contenido
%----------------------------------------------------------------------------------
\begin{document}
%\begin{CJK*}{UTF8}{gbsn}                     % para redactar el CV en chino usando CJK
\maketitle

\section{Education}
\cventry{2015--2019}{B.S. Computer \& Information Science}{University of Oregon}{Eugene, OR}{\textit{3.81 in Major}}{Minor in Mathematics.  Focuses include Parallel Computing, Compiler Construction,\newline Computer Networking, and Artificial Intelligence.}  % Los argumentos del 3 al 6 pueden permanecer vacios
%\cventry{a\~no--a\~no}{Grado}{Instituci\'on}{Ciudad}{\textit{Grade}}{Descripci\'on}

%\section{Tesis de maestr\'ia}
%\cvitem{t\'itulo}{\emph{T\'itulo}}
%\cvitem{sinodares}{Sinodales}
%\cvitem{descripci\'on}{Una breve descripci\'on de la tesis}

\section{Projects}

%\subsection{Computing}

\cventry{}{Full Stack Developer}{Tournament of Tournaments App}{San Francisco Bay Area}{}{I architected, developed, published, and maintain a distributed system for handling a Tournament of Tournaments.  This system consists of a database, a server program, a REST API, an RPC API, a web app, an Android application, and an iOS application.\newline{}%
Achievements:%
\begin{itemize}%
\item Optimized performance and network usage to minimize cost on AWS EC2 and S3.
\item Developed a high performance server implemented 100\% in ES6 leverageing Node.js and Express.js.
\item Exercised dozens of core computer science concepts such as novel algorithm design and analysis, creation of state machines, usage of data structures, usage of graph theory, etc.
\item Published applications on the Google Play Store and iOS App Store.  Continually process bug reports and interact with Users.  Continuously integrate, develop, and deploy fixes and new features.
%   \begin{itemize}%
%   \item Sub-logro (a);
%   \item Sub-logro (b), con sub-sub-logros (¡evite hacer esto!);
%     \begin{itemize}
%     \item Sub-sub-logro i;
%     \item Sub-sub-logro ii;
%     \item Sub-sub-logro iii;
%     \end{itemize}
%   \item Sub-logro (c);
%   \end{itemize}
% \item Logro 3.
\end{itemize}}


\section{Work}
\subsection{Computing}

\cventry{2018--2019}{Undergraduate Researcher}{CDUX Research Group}{University of Oregon}{}{Conducted research into high performance computing on behalf of the Department of Energy and the National Science Foundation.\newline{}%
Achievements:%
\begin{itemize}%
\item Developed novel algorithms leveraging data parallel primitives in collaboration with doctoral students.
\item Ported and parallelized serial code for the Visualization Toolkit (VTK-m).
%   \begin{itemize}%
%   \item Sub-logro (a);
%   \item Sub-logro (b), con sub-sub-logros (¡evite hacer esto!);
%     \begin{itemize}
%     \item Sub-sub-logro i;
%     \item Sub-sub-logro ii;
%     \item Sub-sub-logro iii;
%     \end{itemize}
%   \item Sub-logro (c);
%   \end{itemize}
% \item Logro 3.
\end{itemize}}

\cventry{2018}{Learning Assistant}{Department of Computer and Information Science}{University of Oregon}{}{Provided supplementary education and personal tutoring to a class of 150 2nd year computer science students.\newline{}%
Achievements:%
\begin{itemize}%
\item Taught debugging techniques 15 hours weekly.
\item Used collaborative learning techniques to communicate elemental data structure and algorithm design.
%   \begin{itemize}%
%   \item Sub-logro (a);
%   \item Sub-logro (b), con sub-sub-logros (¡evite hacer esto!);
%     \begin{itemize}
%     \item Sub-sub-logro i;
%     \item Sub-sub-logro ii;
%     \item Sub-sub-logro iii;
%     \end{itemize}
%   \item Sub-logro (c);
%   \end{itemize}
% \item Logro 3.
\end{itemize}}

\subsection{Miscellaneous}
\cventry{2017--2020}{Mathematics Tutor}{Self-Employed}{Eugene, OR/San Ramon, CA}{}{Privately tutored Differential and Integral Calculus, Linear Algebra, and Discrete and Combinatorial Mathematics. Worked flexibly with relaying complex information to students in personally tailored methods.}

% \section{Language}
% \cvitemwithcomment{English}{Native Speaker}{Verbal, Written}
% \cvitemwithcomment{Idioma 2}{nivel}{Comentario}
% \cvitemwithcomment{Idioma 3}{nivel}{Comentario}

\section{Skills}
%\cvitem{Python, C, C++, HTML5, CSS3, JavaScript, React.js, Node.js, Git, REGEX}{}
\cvdoubleitem{AWS}{EC2}{S3}{Google Cloud}
\cvdoubleitem{Python}{TensorFlow}{NumPy}{Pandas}
\cvdoubleitem{C}{C++}{Bison}{Lex}
\cvdoubleitem{HTML5}{CSS3}{JavaScript/ES6}{React.js}
\cvdoubleitem{iOS}{Android}{React Native}{Express.js}
\cvdoubleitem{Git}{REGEX}{jQuery}{Node.js}
\cvdoubleitem{UNIX}{SQL}{Puppet}{Swift}
\cvdoubleitem{Data Mining}{Big Data}{Protocols}{Java}
\cvdoubleitem{DevOps}{PostgreSQL}{APIs}{Docker}

\section{Activities}

\cventry{}{University of Oregon ACM International Collegiate Programming \newline Competition Team}{Flagship Team}{}{}{Represented the University of Oregon in the world’s premier programming competition against teams from top universities such as Stanford, University of Washington, and UC Berkeley. Developed strong relationships and fluid teamwork with my copartners.\newline{}}%

\cventry{}{University of Oregon ARLISS}{Autonomous Drone Competition Team}{}{}{Represented the University of Oregon at ARLISS competition in Black Rock Desert, Nevada. Built an autonomously driving drone which could withstand a rocket launch and ejection at 10,000ft. Constructed a rocket to earn Level 1 and Level 2 HPR Certification from the National Association of Rocketry.\newline{}}%

\cventry{}{University of Oregon Cybersecurity Club}{Member}{}{}{Studied computer vulnerabilities in the context of protecting one’s digital assets. Disseminated information to students and faculty about personal cybersecurity.\newline{}}%

\cventry{}{Oregon Blockchain Group}{Member}{}{}{Studied the systems enabled by blockchain technology in the context of enterprise. Attended seminars conducted by industry leaders in blockchain application, such as Ripple, Public Market, GE, and Intel.\newline{}}%

\cventry{}{University of Oregon Makers Club}{Member}{}{}{Learned and practiced skills such as soldering, using powered drills, rotary tools, and saws, 3D modelling and printing, and programming microcontrollers.\newline{}}%

\bigbreak
\bigbreak
\bigbreak

\section{Awards}
\cvitem{}{2018 Phillip Seeley Scholarship in Computer and Information Science}
\cvitem{}{2019 Luks Programming Competition - 1st Place}
\cvitem{}{2017 Juilfs Programming Competition - 1st Place}
\cvitem{}{2018 Luks Programming Competition - 3rd Place}

\section{Hobbies}
\cvdoubleitem{Ultimate Frisbee}{Disc Golf}{Backpacking}{Music Production}

% \cvitem{University of Oregon ACM International Collegiate Programming Competition Team - Flagship Team}{Represented the University of Oregon in the world’s premier programming competition against teams from top universities such as Stanford, University of Washington, and UC Berkeley. Developed strong relationships and fluid teamwork with my copartners.}
% \cvitem{hobby 2}{Descripci\'on}
% \cvitem{hobby 3}{Descripci\'on}

% \section{Extra 1}
% \cvlistitem{Tema 1}
% \cvlistitem{Tema 2}
% \cvlistitem{Tema 3}

% \renewcommand{\listitemsymbol}{-~}            % para cambiar el simbolo para las listas

% \section{Extra 2}
% \cvlistdoubleitem{Tema 1}{Tema 4}
% \cvlistdoubleitem{Tema 2}{Tema 5\cite{book1}}
% \cvlistdoubleitem{Tema 3}{}

% Las publicaciones tomadas de un archivo de BibTeX sin usar multibib\renewcommand*{\bibliographyitemlabel}{\@biblabel{\arabic{enumiv}}}

\nocite{*}
\bibliographystyle{plain}
\bibliography{publications}                   % 'publications' es el nombre del archivo BibTeX

% Las publicaciones tomadas de un archivo BibTeX usando el paquete multibib
%\section{Publicaciones}
%\nocitebook{book1,book2}
%\bibliographystylebook{plain}
%\bibliographybook{publications}              % 'publications' es el nombre del archivo BibTeX
%\nocitemisc{misc1,misc2,misc3}
%\bibliographystylemisc{plain}
%\bibliographymisc{publications}              % 'publications' es el nombre del archivo BibTeX

%\clearpage\end{CJK*}                          % si esta redactando su CV en chino usando CJK, \clearpage es requerido por fancyhdr para que funcione correctamente con CJK, aunque esto eliminara la numeracion de pagina al dejar \lastpage como no definido
\end{document}


%% fin del archivo `template-es.tex'.
